\documentclass{article}
\usepackage[utf8]{inputenc}

\usepackage{amsmath, amssymb, amsthm}
\usepackage{upgreek}
\usepackage{physics}
\usepackage[version=4]{mhchem}
\usepackage{braket}
\usepackage{esvect}

% \graphicspath{{figure/}}
\usepackage{booktabs}
\usepackage{multirow}

\newcommand{\reftab}[1]{\textbf{Table~\ref{#1}}}
\newcommand{\reffig}[1]{\textbf{Figure~\ref{#1}}}
\newcommand{\refeq}[1]{\textbf{Equation~(\ref{#1})}}

\title{DOF problem}
\author{Henglu Xu}
\date{\today}

\begin{document}
\section{Degree of freedom of a system with external forces}

From MD simulations with different $N$(\ce{CH4}),
we compute average kinetic energy $E_k$ of \ce{CH4} as listed in \reftab{tab:Ek_nCH4}.

\begin{table}[htbp]
    \centering
    \caption{Average $E_k$ of \ce{CH4} for different $N$(\ce{CH4})}
    \label{tab:Ek_nCH4}
    %in MD tests of $n$(\ce{CH4})
    \begin{tabular}{ccclcl}
    \toprule
    $N$%(\ce{CH4})
    & $\frac{E_k}{N}$~(kJ/mol)\\
    % & percentage error\\
    \midrule
     4  & 2.78 \\   %& -25.0\% \\
     8  & 3.23 \\   %& -12.6\% \\
    16  & 3.48 \\   %&  -6.0\% \\
    40  & 3.61 \\   %&  -2.4\% \\
    80  & 3.66 \\   %&  -1.1\% \\
    100 & 3.67 \\   %& -0.74\% \\
    120 & 3.67 \\   %& -0.72\% \\
    140 & 3.68 \\   %& -0.73\% \\
    160 & 3.68 \\   %& -0.45\% \\
    \bottomrule
    \end{tabular}
\end{table}

According to the equipartition theorem,
energy carried from every degree of freedom~(DOF) is equal to $\frac{1}{2}k_\text{B} T$,
and the sum of the energies for all DOF ($N_\text{DOF}\times \frac{1}{2}k_\text{B} T$), is equal to total kinetic energy $E_k$.
From this discussion, we obtain the relation between total kinetic energy $E_k$ and temperature $T$ as \refeq{eq:dof_thm}.
To calculate total DOF ($N_\text{DOF}$) of \ce{CH4} group, we assume it has the form of "$N_\text{DOF}=aN-b$" and perform a linear regression with data of $E_k$ and $N$ from \reftab{tab:Ek_nCH4}, $T$=298~K,
$N_\text{DOF}$ is equal to $3N-3$ (\refeq{eq:dof_value}).

\begin{align}
    \label{eq:dof_thm}
    &E_k=\sum^i_N \frac{1}{2}mv^2_i
    =\frac{1}{2} k_\text{B} T \times N_\text{DOF}\\
    \label{eq:dof_value}
    &N_\text{DOF}
    =3.01\times N-3.00
\end{align}

This result supports the idea that the calculation of degrees of freedom in LAMMPS is implemented in a way which keeps total linear momentum conserved and center of mass (COM) fixed,
despite of the existence of an external force field.
This correction of DOF has an effect on the velocities of \ce{CH4}, and, consequently, on the $E_k$ distribution.

The conservation of total linear momentum determines total DOF of \ce{CH4} is equal to $3N-3$ instead of $3N$, which cause the anomalous relation.
Indeed, to use the correct DOF estimation, we modify the source codes of LAMMPS related to the DOF calculations in the NVT simulation.

\begin{verbatim}
Lammps-version/src/:
    fix_nvt.cpp         \*call fix_nh.cpp*\
    fix_nh.cpp:         \*call "dof" variable from compute_temp.cpp*\
                        tdof=temperature->dof
                        kecurrent=tdof*boltz*t_current
    compute.cpp:        \*set default value of "extra_dof"*\
                                \*and "fix_dof"*\
                        extra_dof=domain->dimension
                        fix_dof=0
    compute_temp.cpp:   \*definition of "dof" variable*\
                        dof=domain->dimension*natoms_temp
                        dof-=extra_dof+fix_dof
\end{verbatim}

For using Nose-Hoover thermostat, four \texttt{*.cpp} files are the most relevant.
\texttt{fix\_nvt.cpp} and \texttt{fix\_nh.cpp} include all implementations of Nose-Hoover thermostat, and refer to variable \texttt{dof} to compute $E_k$.
\texttt{compute.cpp} and \texttt{compute\_*.cpp} include implementations of computation of certain property, e.g.\ \texttt{compute\_temp.cpp} for temperature calculation.

In \texttt{compute\_temp.cpp}, we find the definition of $N$(\ce{CH4}) for a 3-dimensional system,
\texttt{dof=3N-extra\_dof-fix\_dof}.
In \texttt{compute.cpp}, default value of \texttt{extra\_dof} is assigned to be equal to dimension, i.e.\ 3 for 3-dimensional system.

In conclusion, LAMMPS assigns the total DOF of our system to be ($3N-3$) instead of $3N$, which causes the lower total kinetic energy $E_k$ compared to the reference value, 3.70 kJ/mol (\refeq{eq:3.7}), and incorrect $E_k$ distribution compared to the reference canonical distribution (\refeq{eq:PvsE}).

\begin{align}
    &\textbf{Canonical distribution at 298~K} \nonumber \\
    \label{eq:3.7}
    &E_k~\text{per mol}
    =\frac{3}{2}k_\text{B} T N_\text{A}=3.70~\text{kJ/mol}\\%\quad\text{for 298~K}\\
    \label{eq:PvsE}
    &P=2\left(\frac{E}{\pi}\right)^{\frac{1}{2}} \left(\frac{1}{k_BT}\right)^\frac{3}{2} \exp{-\frac{E}{k_BT}}
\end{align}

In other words, when we perform a NVT simulation at 298~K using LAMMPS, the target temperature is never achieved,
more significant for small numbers of guest molecules, due to loss of energies of 3 DOF,
and, in addition, the effective temperature of our simulation is $T_\text{eff}=\frac{N-1}{N}\times 298~\text{K}$.
%When $N$ is smaller than 40~\ce{CH4},
The effective temperature $T_\text{eff}$ is lower than the imposed one, leading to slower diffusion of \ce{CH4} and smaller diffusion coefficients $D_s$, which explains the anomalous relation between $N$(\ce{CH4}) and $D_s$.

To obtain the correct results of $D_s$ at 298~K, we modify \texttt{compute\_temp.cpp},
use \texttt{dof=3N-fix\_dof}
instead of \texttt{dof=3N-extra\_dof-fix\_dof}.
By removing \texttt{extra\_dof} variable, LAMMPS is able to calculate the total DOF of \ce{CH4} as $3N$.

The modifed Lammps is available in
`https://github.com/HengluXu/lammps.git`
%
\end{document}
